%!/usr/bin/env pdflatex
%-*- coding: utf-8 -*-
%@author : Romain Graux
%@date : 2022 May 03, 12:22:56
%@last modified : 2022 May 04, 18:11:46

As we produce more data every year, we need to find a way to store it efficiently. Currently, we store our data in big data center consuming a lot of energy to keep these informations in electronic devices, \illustrate{Find the consumption of data centers}

It would be a good idea to find a way to store our data in a more efficient and ecological way. For storing information, hard drives don't hold a candle to DNA. Our genetic code packs billions of gigabytes into a single gram. A mere milligram of the molecule could encode the complete text of every book in the Library of Congress and have plenty of room to spare. \cite{bib:dna_data_storage}

% Since it is now possible to manipulate synthetic DNA \illustrate{Find an article talking about the first DNA synthesize}, it would be one of a new type 

But it can not be applied to all data types, for example, it is not possible yet to replace an USB stick by a DNA based USB stick and expecting the same experience. The information retrieval latency and high cost of the DNA sequencer and other instruments "currently makes this impractical for general use," says Daniel Gibson, a synthetic biologist at the J. Craig Venter Institute in Rockville, Maryland, "but the field is moving fast and the technology will soon be cheaper, faster, and smaller." Gibson led the team that created the first completely synthetic genome, which included a "watermark" of extra data encoded into the DNA. \cite{bib:dna_data_storage}

This does not mean that there are no applications for DNA based storage. DNA based storage can be used for long term media preservation archives (so called cold media storage) which are infrequently accessed and thus do not need low information retrieval latency.


\subsection{Constraints}

Unfortunately, nucleic acids have biological constraints and can not be assembled in any order like it is the case for binary digits. The DNA strands have to be created in a way that the double helix binds well together and is not immediately desctructed. We must therefore respect the biological constraints to build strong strands that can last for a certain period of time. 

In this part, we are going to go through some of the constraints that we have to respect to build a DNA strand. Unfortunately, the list of constraints is not exhaustive and in the real life, each arangement of nucleic acids has an impact on the strength of a strand, therefore we can only simulate the longevity but not strictly respect the constraints. 

\tocontinue{}

\subsection{Requirements}

\tocontinue{}


