%!/usr/bin/env pdflatex
%-*- coding: utf-8 -*-
%@author : Romain Graux
%@date : 2022 June 07, 11:28:34
%@last modified : 2022 June 17, 19:23:52

After observing all the results, we can say that we have indeed built a working end-to-end point cloud compressor for DNA based storage capable of producing DNA stream sequences and recover the point clouds from them. 
We can also say that the end-to-end point cloud compressor is adaptive, the quality over rate can be tuned by the user as seen on the rate-distortion tradeoff curve. 

The model is one of the first to be used in the field of point cloud compression for DNA storage and thus is a good starting point for future work but can not be considered as a complete and optimal solution since it uses several implementations that are not initially built for our purpose.  

Improvements have to be done on the nucleotide rate and the achievable reconstruction distortion with the highest quality parameters. 
Improvements have also to be done on the robustness of the model and the ability to handle noise in the compressed domain.
These improvements can be done either by entirely rebuilding the model architecture or by adapting the different model parts to our needs.

As previsouly said, this is one of the first model in the world that compress point clouds into DNA sequences.
It therefore opens the door to this interesting and topical subject and can serve as a baseline to compare future models.
